\chapter*{课程安排}

在短短的6次课中,我们将对统计力学/统计热力学的内容做一个简单的介绍,主要帮助大家建立理论的框架,其代价是例子较少。

安排如下:
\begin{itemize}
    \item Statistical Mechanics 的基本假设;孤立系的平衡条件;热力学四定律的导出
    \item 正则系综;正则配分函数;配分函数的分解
          \begin{equation*}
              Q \implies q^N \text{ or } \frac{q^N}{N!} \implies q = q_t q_n q_e q_v q_r
          \end{equation*}
    \item 例子
          \begin{itemize}
              \item 无相互作用的单原子气体$(q=q_t)$
                    \begin{equation*}
                        q=q_t \implies Q \implies 
                            \begin{cases}
                                A \implies p \implies PV=nRT \\
                                U \implies C_v = \frac{3}{2}Nk
                            \end{cases}
                    \end{equation*}
              \item 同上,经典处理:Gibbs佯谬;“状态”的意义
              \item 经典的振子$(q=q_v)$:能均分定理及其推论,局限性
              \item Maxwell 速度分布律的导出
          \end{itemize}
    \item 量子化学计算求气相反应的热力学数据,例如
\end{itemize}



\chapter{统计力学的基本假设}
统计力学假设很少,而且相当简单。统计力学假设:


\begin{postulate}
对孤立系统,每一可能的状态都具有相同的概率。即
\begin{equation}
    \mathbb{P}(j) = \left\{
        \begin{aligned}
            1/\Omega &, \text{ if } E_j = E \\
            0 &, \text{ if } E_j \neq E
        \end{aligned}
    \right.
\end{equation}
\end{postulate}

\begin{rek}
    该假设中隐含了“各态历经”的假设。
\end{rek}

\begin{rek}
    在大数极限下,我们只能观察到最可几的态。
\end{rek}

\begin{equation}
    S = k \ln \Omega
\end{equation}
\chapter*{课程安排}

在短短的6次课中,我们将对统计力学/统计热力学的内容做一个简单的介绍,主要帮助大家建立理论的框架,其代价是例子较少。

安排如下:
\begin{itemize}
    \item 
    \item 2
    \item 3
    \item 4
    \item 5
    \item 6
    \item 
\end{itemize}



\chapter{统计力学的基本假设}
统计力学假设很少,而且相当简单。统计力学假设:


\begin{hypo}
对孤立系统,每一可能的状态都具有相同的概率。即
\begin{equation}
    \mathbb{P}(j) = \left\{
        \begin{aligned}
            1/\Omega &, \text{ if } E_j = E \\
            0 &, \text{ if } E_j \neq E
        \end{aligned}
    \right.
\end{equation}
\end{hypo}

\begin{rek}
    该假设中隐含了“各态历经”的假设。
\end{rek}

\begin{rek}
    在大数极限下,我们只能观察到最可几的态。
\end{rek}

\begin{equation}
    S = k \ln \Omega
\end{equation}
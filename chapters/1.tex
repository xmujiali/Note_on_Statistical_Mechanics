\chapter*{课程安排}

在短短的6次课中,我们将对统计力学/统计热力学的内容做一个简单的介绍,主要帮助大家建立理论的框架,其代价是例子较少。

安排如下:
\begin{itemize}
    \item Statistical Mechanics 的基本假设;孤立系的平衡条件;热力学四定律的导出
    \item 正则系综;正则配分函数;配分函数的分解
          \begin{equation*}
              Q \implies q^N \text{ or } \frac{q^N}{N!} \implies q = q_t q_n q_e q_v q_r
          \end{equation*}
    \item 例子
          \begin{itemize}
              \item 无相互作用的单原子气体$(q=q_t)$
                    \begin{equation*}
                        q=q_t \implies Q \implies
                        \begin{cases}
                            A \implies p \implies PV=nRT \\
                            U \implies C_v = \frac{3}{2}Nk
                        \end{cases}
                    \end{equation*}
              \item 同上,经典处理:Gibbs佯谬;“状态”的意义
              \item 经典的振子$(q=q_v)$:能均分定理及其推论,局限性
              \item Maxwell 速度分布律的导出
          \end{itemize}
    \item 量子化学计算求气相反应的热力学数据,例如
\end{itemize}



\chapter{统计力学的基本假设}
统计力学假设很少,而且相当简单。统计力学假设:


\begin{postulate}
    对孤立系统,每一可能的状态都具有相同的概率。即
    \begin{equation}
        \mathbb{P}(j) = \left\{
        \begin{aligned}
            1/\Omega & , \text{ if } E_j = E    \\
            0        & , \text{ if } E_j \neq E
        \end{aligned}
        \right.
    \end{equation}
\end{postulate}

\begin{rek}
    该假设中隐含了“各态历经”的假设。
\end{rek}

\begin{rek}
    一个孤立系统的全部微观状态,可以对应于不同的宏观态。然而,在大数极限下,我们只能观察到最可几的宏观态,其余宏观态根本无法观察到。谁见过10000个硬币抛掷的结果全部为正呢?
\end{rek}

\begin{defn}
    统计熵$S$,定义为:
    \begin{equation}
        S := k \ln \Omega
    \end{equation}
\end{defn}
\begin{rek}
    我们定义的$S$随$\Omega$单调增加,故两者可以一一对应,最可几的$\Omega$等价于最大的$S$。比例系数$k$的选择是为了方便起见,稍后我们将看到这样定义的统计熵和热力学熵是一致的。
\end{rek}

\chapter{孤立系平衡的条件;热力学四定律的导出}
\section{热平衡的条件}
将一个系统用\emph{固定的、导热的、不可透过粒子}的壁分为两部分。则
\begin{align*}
    E_1 + E_2 & = E_0 \\
    V_1 + V_2 & = E_0 \\
    N_1 + N_2 & = E_0
\end{align*}

\begin{align*}
    \Omega_1 & = \Omega_1(E_1, V_1, N_1) = \Omega_1(E_1) \\
    \Omega_2 & = \Omega_2(E_2, V_2, N_2) = \Omega_2(E_2) \text{, where } E_2 = E_0 - E_1 \\
    \Omega   & = \Omega_1(E_1)\Omega_2(E_2) \\
    \frac{\partial \Omega}{\partial E_1} 
    & = \frac{\partial \Omega_1}{\partial E_1} \Omega_2 
    + \Omega_1 \frac{\partial \Omega_2}{\partial E_2} \frac{\mathrm{d} E_2}{\mathrm{d} E_1} \\
    & = \frac{\partial \Omega_1}{\partial E_1} \Omega_2 - \Omega_1 \frac{\partial \Omega_2}{\partial E_2}
\end{align*}
两边同时除以$\Omega = \Omega_1 \Omega_2$,再利用$\frac{1}{y}\frac{\partial y}{\partial x} = \frac{\partial \ln y}{\partial x}$,得到:
\begin{equation*}
    \frac{\partial S}{\partial E_1} = \frac{\partial S_1}{\partial E_1} - \frac{\partial S_2}{\partial E_2}
\end{equation*}
由于$S$不减,$\mathrm{d}E_1 > 0 \iff \frac{\partial S_1}{\partial E_1} - \frac{\partial S_2}{\partial E_2} > 0 \iff \frac{\partial S_1}{\partial E_1} > \frac{\partial S_2}{\partial E_2}$,反之亦然。可见,热量从$\frac{\partial S}{\partial E}$较低处流向$\frac{\partial S}{\partial E}$较高处,$\frac{\partial S}{\partial E}$与温度究竟时何关系呢?
\begin{defn}
    $\frac{1}{T} := \left(\frac{\partial S}{\partial E}\right)_{N,V}$
\end{defn}
根据上面的讨论,热平衡的条件是$T_1 = T_2$。

回忆热力学中,$\mathrm{d}E = T\mathrm{d}S - p\mathrm{d}V + \mu \mathrm{d}N$,即
\begin{equation*}
    \mathrm{d}S = \frac{1}{T} \mathrm{d}E + \frac{p}{T}\mathrm{d}V - \frac{\mu}{T}\mathrm{d}N
\end{equation*}
在我们所考虑的情形下,体积和粒子数都不会变化,后两项不存在。对比发现,我们定义的$T$与热力学温度是一致的。


\section{压力平衡的条件}
将一个系统用\emph{可动的、导热的、不可透过粒子}的壁分为两部分。


\section{相/化学平衡的条件}
将一个系统用\emph{固定的、导热的、可透过粒子}的壁分为两部分。

